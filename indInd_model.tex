\documentclass{article}

\usepackage{proof}
\usepackage{mathtools}
\usepackage{amsthm}
\usepackage{amssymb}
\usepackage{amsmath}

\usepackage{MnSymbol}
\usepackage{stmaryrd}

\usepackage{enumerate}
\usepackage{microtype}

\usepackage{hyperref}

%\title{A set-theoretic model of induction-induction}
\title{A finite axiomatisation of inductive-inductive definitions}

\usepackage{authblk}
\author{Fredrik Nordvall Forsberg}
\author{Anton Setzer}
\affil{Swansea University \\
Singleton  Park \\
Swansea SA2 8PP, UK}

%\thanks{Supported by EPSRC grant EP/G033374/1, Theory and applications of induction-recursion.}}

\input macros.tex

\begin{document}

\maketitle

\begin{abstract}  \noindent %todo: rewrite abstract to reflect new focus of article
  Induction-induction is a principle for mutually defining data types
  $A : \Set$, $B : A \to \Set$. Both $A$ and $B$ are defined
  inductively, and the constructors for $A$ can refer to $B$ and vice
  versa. In addition, the constructor for $B$ can refer to \emph{the
    constructor} for $A$. Induction induction occurs in a natural way
  when formalising dependent type theory in type theory.  We give some
  examples of inductive-inductive definitions, such as the set of
  surreal numbers. We then formulate the principle of
  induction-induction, and prove its consistency by constructing a
  model.
%Finally we analyse the relationship between induction-induction and indexed 
%inductive definitions.
\end{abstract}

\section{Introduction} %todo: rewrite introduction
\label{sec:introduction}

%Martin-L\"of Type Theory \cite{martinlof1984bibliopolis} is a
%foundational framework for constructive mathematics, where induction
%plays a major part in the construction of sets.

Induction is an important principle of definition and reasoning,
especially so in constructive mathematics.  Martin-L\"of's
formulation~\cite{martinlof1984bibliopolis} of type theory, for
instance, includes inductive definitions of among others Cartesian
products, disjoint unions, the natural numbers, well-orderings and
lists. Martin-L\"of also includes the identity set and the family of
finite sets, but these are examples not of ordinary inductive
definitions, but of inductive families~\cite{dybjer1994indfam}, where
a family $X : I \to \Set$ for some fixed index set $I$ is defined
inductively simultaneously.

One of the most important contributions of
Martin-L\"of~\cite{martinlof1984bibliopolis} is the definition of the
universe \`a la Tarski~\cite{palmgren1998twentyfiveyears}, where a set
$U$ (thought of as containing codes for sets) is defined inductively
together with a function $T : U \to \Set$ (thought of as mapping a
code $u : U$ to the set $T(u)$ it represents) which is defined
recursively. This is an instance of the principle of
inductive-recursive definitions~\cite{dybjer2000IR}. The constructors
for $U$ may depend negatively on $T$ applied to elements of $U$, as is
the case if $U$ for example is closed under dependent function spaces:
\[
\infer{\pi(a, b) : U}{a : U & \quad b : T(a) \to U}
\]
with $T(\pi(a, b)) = (x : T(x)) \to T(b(x))$.

Here, $T : U \to \Set$ is defined recursively. Sometimes, however, one
might not want to give $T(u)$ completely as soon as $u : U$ is
introduced, but instead define $T$ inductively as well.

In this article, we present another induction principle, which we, in
reference to induction-recursion, call \emph{induction-induction}. A set $A$ is
inductively defined simultaneously with an $A$-indexed set $B$, which is also
inductively defined, and the introduction rules for $A$ may also refer to
$B$. So we have formation rules $A : \Set$, $B : A \to \Set$ and typical
introduction rules might take the form
\[ \infer{\text{intro}_A(a, b, \ldots) : A}{a : A &b : B(a) & \ldots} \quad
\infer{\text{intro}_B(a_0, b, a_1, \ldots) : B(a_1)}{a_0 : A &
&b : B(a_0) & a_1 : A & \ldots} \]

This is not a simple mutual inductive definition of two sets, as $B$
is indexed by $A$. It is not an ordinary inductive family, as $A$ may
refer to $B$. Finally, it is not an instance of induction-recursion,
as $B$ is constructed inductively, not recursively.

$<$Schemas and universes of descriptions$>$.

%External schemas for general inductive sets and inductive families
%have been given by Backhouse et.\ al.\ \cite{backhouse1989diytt} %backhouse1988meaning
%and Dybjer \cite{dybjer1994indfam} respectively.

%this has been internalised by Dybjer and Setzer
%\cite{dybjersetzer1999finax,dybjersetzer2003inalg,dybjersetzer2006IIR}.

%Inductive-recursive
%definitions have also been used for generic programming in dependent
%type theory \cite{benke2003universes}.



\subsection{Examples of inductive-inductive definitions}
\label{sec:examples}

\begin{example}[Contexts and types]
\label{ex:ctxt-type}


Danielsson \cite{danielsson2007formalisation} and Chapman
\cite{chapman2009eatitself} define the syntax of dependent type theory
in the theory itself by inductively defining contexts, types in a
given context and terms of a given type. Let us concentrate on
contexts and types for simplicity. There should be an empty
context $\emptyCtxt$, and if we have any context $\Gamma$ and a valid
type $\sigma$ in that context, then we should be able to extend the
context with a fresh variable of that type. We end up with the
following inductive definition of the set of contexts:
\[
\infer{\emptyCtxt : \Ctxt}{} \qquad
\infer{\consCtxt{\Gamma}{\sigma} : \Ctxt}{\Gamma : \Ctxt & \sigma : \Ty(\Gamma)}
\]

For types, let us have a base type $\baseTy{}$ (valid in any context)
and a dependent function type: if $\sigma$ is a type in context $\Gamma$,
and $\tau$ is a type in $\Gamma$ extended with a fresh variable of type
$\sigma$ (the variable from the domain), then $\Pi(\sigma, \tau)$ is a
type in the original context. This leads us to the following inductive
definition of $\Ty : \Ctxt \to \Set$:
\[
\infer{\baseTy{\Gamma} : \Ty(\Gamma)}{\Gamma : \Ctxt} \qquad
\infer{\piTy{\Gamma}{\sigma}{\tau} : \Ty(\Gamma)}{\Gamma : \Ctxt
                    & \sigma : \Ty(\Gamma)
                    & \tau : \Ty(\consCtxt{\Gamma}{\sigma})}
\]

Note that the definition of $\Ctxt$ refers to $\Ty$, so both sets
have to be defined simultaneously. Another peculiarity is how the
introduction rule for $\Pi$ explicitly focuses on a specific
constructor in the index of the type of $\tau$.
\blackqed

% Inductive-inductive definitions have been used by Dybjer~\cite{dybjer1996internalTT}, Danielsson~\cite{danielsson2007formalisation} and
% Chapman~\cite{chapman2009eatitself} to internalise the syntax and semantics of
% type theory. \label{ctxtsandtypes} 
% Slightly simplified, they define a set $\Ctxt$ of contexts, a
% family $\Ty : \Ctxt \to \Set$ of types in a given context, and a family
% $\Term : (\Gamma : \Ctxt) \to \Ty(\Gamma) \to \Set$ of terms of a
% given type. Let us for simplicity only consider contexts and
% types. The set $\Ctxt$ of contexts has two constructors
% \begin{align*}
%   \nil &: \Ctxt \enspace ,\\
%   \cons &: (\Gamma : \Ctxt) \times \Ty(\Gamma) \to \Ctxt\enspace ,
% \end{align*}
% corresponding to the empty context and extending a context $\Gamma$
% with a new type. In our simplified setting, $\Ty : \Ctxt \to \Set$ has the following constructors
% \begin{align*}
%   \setz &: (\Gamma : \Ctxt) \to \Ty(\Gamma) \enspace ,\\
%   \Pi &:  (\Gamma : \Ctxt) \times (A : \Ty(\Gamma)) \times \Ty(\cons(\Gamma, A)) \to \Ty(\Gamma)\enspace . %\\
% %  \El &:  (\Gamma : \Ctxt) \to \Term (\Gamma, \setz) \to \Ty(\Gamma).
% \end{align*}
% The first constructor states that $\setz$ is a type in any
% context. The second constructor $\Pi$ is the constructor for the $\Pi$-type: If we have a
% type $A$ in a context $\Gamma$, and another type $B$ in $\Gamma$
% extended by $A$ (corresponding to abstracting a variable of type $A$), then
% $\Pi(A, B)$ is also a type in $\Gamma$. 
% %Finally, $\El$ states that every term of type $\Set$ is also a type.

% Note how the constructor $\cons$ for $\Ctxt$ has an argument of type
% $\Ty(\Gamma)$, even though $\Ty$ is indexed by $\Ctxt$. It is also worth
% noting that $\Pi$ has an argument of type $\Ty(\cons(\Gamma, A))$, i.e.\ we are
% using the constructor for $\Ctxt$ in the index of $\Ty$. In general, we could
% of course imagine an argument of type $\Ty(\cons(\cons(\Gamma, A), A'))$ etc.
\end{example}

\begin{example}[Sorted lists]
\label{ex:sorted-list}

Let us define a data type consisting of sorted lists (of natural
numbers, say). With induction-induction, we can simultaneously define
the set $\SortedList$ of sorted lists and the predicate $\lessList :
(\Nat \times \SortedList) \to \Set$ with $n \lessList \ell$ true if
$n$ is less than or equal to every element of $\ell$.

The empty list is certainly sorted, and if we have a proof $p$ that
$n$ is less than or equal to every element of the list $\ell$, we can
put $n$ in front of $\ell$ to get a new sorted list
$\consList{n}{\ell}{p}$. Translated into introduction rules, this becomes:
\[
\infer{\nilList : \SortedList}{} \qquad
\infer{\consList{n}{\ell}{p} : \SortedList}{n : \Nat \quad & \ell : \SortedList \quad & p : n \lessList \ell}
\]
For $\lessList$, we have that every $m : \Nat$ is trivially smaller
than every element of the empty list, and if $m \leq n$ and
inductively $m \lessList \ell$, then $m \lessList \consList{n}{\ell}{p}$:
\[
\infer{\nilLess{m} : m \lessList \nilList}
%{m : \Nat}
{} \qquad
\infer{\consLess{n}{\ell}{p}{m}{q}{p_{m, \ell}}\ : m \lessList \consList{n}{\ell}{p}}
%      {m, n : \Nat \quad & \ell : \SortedList \quad & p : n \lessList \ell \quad &
       {q : m \leq n \quad & p_{m, \ell} : m \lessList \ell}
\]
%
% For an example with more of a programming flavour, consider defining a
% data type consisting of sorted lists (of natural numbers, say). With
% induction-induction, we can simultaneously define the set
% $\SortedList$ of sorted lists and the predicate $\lessList : (\Nat
% \times \SortedList) \to \Set$ with $n \lessList \ell$ true if $n$ is
% less than or equal to every element of $\ell$.

% The empty list is certainly sorted, and if we have a proof $p$ that
% $n$ is less than or equal to every element of the list $\ell$, we can
% put $n$ in front of $\ell$ to get a new sorted list
% $\consList{n}{\ell}{p}$. Translated into introduction rules, this becomes:
% \[
% \infer{\nilList : \SortedList}{} \qquad
% \infer{\consList{n}{\ell}{p} : \SortedList}{n : \Nat \quad & \ell : \SortedList \quad & p : n \lessList \ell}
% \]
% For $\lessList$, we have that every $m : \Nat$ is trivially smaller
% than every element of the empty list, and if $m \leq n$ and
% inductively $m \lessList \ell$, then $m \lessList \consList{n}{\ell}{p}$:
% \[
% \infer{\nilLess{m} : m \lessList \nilList}
% %{m : \Nat}
% {} \qquad
% \infer{\consLess{n}{\ell}{p}{m}{q}{p_{m, \ell}}\ : m \lessList \consList{n}{\ell}{p}}
% %      {m, n : \Nat \quad & \ell : \SortedList \quad & p : n \lessList \ell \quad &
%        {q : m \leq n \quad & p_{m, \ell} : m \lessList \ell}
% \]

% Of course, there are many alternative ways to define such a data type
% using ordinary induction, but the inductive-inductive one seems
% natural and might be more convenient for some purposes.
\blackqed
\end{example}

\begin{example}[Conway's surreal numbers]
\label{ex:surreal}

Conway \cite{conway2001ONAG} informally uses induction-induction (but
couched in ZF set theory, not type theory) in order to define his
\emph{surreal numbers}. The ``set''\footnote{There are some size
  issues, since the surreal numbers contain the class of ordinals.} of
surreal numbers is defined inductively, together with an order
relation on surreal numbers which is also defined inductively:

\begin{itemize}
\item A surreal number $X = (X_\mathrm{L}, X_\mathrm{R})$ consists of
  two sets $X_\mathrm{L}$ and $X_\mathrm{L}$ of surreal numbers, such
  that no element from $X_\mathrm{L}$ is greater than any element from
  $X_\mathrm{R}$.
\item A surreal number $Y = (Y_\mathrm{L}, Y_\mathrm{R})$ is greater
  than another surreal number $X = (X_\mathrm{L}, X_\mathrm{R})$, $X \leq Y$, if and
  only if
  \begin{itemize}
  \item there is no $x \in X_\mathrm{L}$ such that $Y \leq x$, and
  \item there is no $y \in Y_\mathrm{R}$ such that $y \leq X$.
  \end{itemize}
\end{itemize}

Both rules can be understood as inductive definitions. However, they
are not formulated with a constructive reading in mind (too many
negations!). Mamane~\cite{mamane2004surrealCoq} develops the theory of
surreal nubmers in the proof assistant Coq~\cite{coq}. %todo: expand
\blackqed
\end{example}


\subsection{Inductive-inductive definitions versus inductive-recursive definitions}
\label{sec:indind-vs-IR}

\section{Type-theoretical preliminaries}
\label{sec:preliminaries}

We work in a type theory with at least two universes $\Set$ and
$\TYPE$, with $\Set : \TYPE$ and $\Set$ a subuniverse of $\TYPE$,
i.e.\ if $A : \Set$ then $A : \TYPE$. Both $\Set$ and $\TYPE$ are
closed under dependent function types, written $(x : A) \to B$, where
$B$ is a set or type depending on $x: A$. Abstraction is written as
$\lambda x : A . e$, where $e : B$ depending on $x : A$, and
application as $f(x)$. Repeated abstraction and application are
written as $\lambda x_1 : A_1 \ldots x_k : A_k . e$ and $f(x_1,
\ldots, x_k)$. If the type of $x$ can be inferred, we simply write
$\lambda x.e$ as an abbreviation.  Furthermore, both $\Set$ and
$\TYPE$ are closed under dependent products, written $(x : A) \times
B$, where $B$ is a set or type depending on $x: A$, with pairs
$\langle a , b \rangle$, where $a : A$ and $b : B[x \coloneqq a]$.
% and projections $\pi_1$, $\pi_2$
We also have $\beta$- and $\eta$-rules for both dependent function types
and products.

We need an empty type $\zero : \Set$, with elimination ${!_A : \zero
  \to A}$ for every $A : \Set$. We need a unit type $\one : \Set$,
with unique element $\oneelt : \one$ and an $\eta$-rule stating that
if $x : \one$, then $x = \oneelt : \one$. Moreover, we include a two
element set $\two : \Set$, with elements $\twott : \two$, $\twoff :
\two$ and elimination constant $\IF~\cdot~\THEN~\cdot~\ELSE~\cdot :$
$(a : \two) \to A(\twott) \to A(\twoff) \to A(a)$ where $i : \two
\Rightarrow A(i) : \TYPE$. I should satisfy the obvious computation
rules, i.e.\ $\IF~\twott~\THEN~a~\ELSE~b = a$ and
$\IF~\twoff~\THEN~a~\ELSE~b = b$. Finally, we require the type $\Nat$
of natural numbers.

With $\IF~\cdot~\THEN~\cdot~\ELSE~\cdot$ and dependent products, we can now
% as in \cite[A.2]{dybjersetzer2006IIR}
define the disjoint union of two sets $A + B \coloneqq$ ${(x : \two)}
\times (\IF~x~\THEN~A~\ELSE~B)$ with constructors $\inl = \lambda {a :
  A} . \langle \twott , a \rangle$ and $\inr = \lambda {b : B}
. \langle \twoff , b \rangle$, and prove the usual formation,
introduction, elimination and equality rules. Importantly, we get
large elimination for sums, since we have large elimination for
$\two$. We write the eliminator $[f, g] : {(c : A + B)} \to C(c)$, where
$i : A + B \Rightarrow A(i) : \TYPE$ and $f : (a : A) \to C(\inl(a))$,
$g : (b : B) \to C(\inr(b))$. The equality rules are
\begin{align*}
[f, g](\inl(a)) &= f(a) \enspace , \\
[f, g](\inr(b)) &= g(b) \enspace .
\end{align*}
We write $\bigplus_{k = 0}^n A_k$ for the iterated sum $A_0 + (A_1 +
(\ldots + A_n)\cdots)$ and $[f_0, f_1, \ldots, f_n]$ for the iterated
case distinction $[f_0, [f_1, [\ldots , f_n]]\cdots]$.
% We write $A_0 + A_1 + \ldots + A_n$ for $A_0 + (A_1 + (\ldots +
% A_n)\cdots)$ and $\inn_k(a)$ for the $k$th injection
% $\inl(\inr^k(a))$ (with special case $\inn_n(a)$ = $\inr^{n}(a)$).

Intensional type theory in Martin-L\"of's logical framework extended with
dependent products and $\zero$, $\one$, $\two$ and $\Nat$ has all the features we
need. Thus, our development can be seen as an extension of
the logical framework.

\section{A finite axiomatisation}
\label{sec:axiomatisation}
 
%TODO: rewrite

In this section, we give a finite axiomatisation of a type theory with
inductive-inductive definitions. This axiomatisation differs slightly
from our previous
axiomatisation~\cite{nordvallforsbergSetzer2010indind}, in that it
hopefully is easier to understand. However, the definable sets should
be the same for both axiomatisations.

%TODO: rewrite

We take inspiration from Dybjer and Setzer's axiomatisation of
inductive-recursive definitions~\cite{dybjersetzer1999finax}. The main
idea is to construct a universe consisting of codes for
inductive-inductive definitions, together with a decoding function,
which maps a code $\varphi$ to the domain of the constructor for the
inductively defined set represented by $\varphi$.  Just as the second
component $B : A \to \Set$ depends on the first set $A$, the universe $\SPB : \SPA \to \TYPE$
of codes for the second component will depend on the universe $\SPA$ of codes
for the first set.

\subsection{Dissecting an inductive-inductive definition}
\label{sec:dissect-ind}

We wish to formalise and internalise an inductive-inductive definition
given by constructors
\[
\introA : \Phi_{\mathrm{A}}(A, B) \to A
\]
and
\[
\introB : (x : \Phi_{\mathrm{B}}(A, B, \introA)) \to B(\theta(x))
\]
for some $\Phi_{\mathrm{A}}(A, B) : \Set$, $\Phi_{\mathrm{B}}(A, B,
\introA) : \Set$ and $\theta : \Phi_{\mathrm{B}}(A, B, \introA) \to
A$.
Here, $\theta(x)$ is the \emph{index} of $\introB(x)$, i.e.\ the element $a :
A$ such that $\introB(x) : B(a)$.

Not all expressions $\Phi_{\mathrm{A}}$ and $\Phi_{\mathrm{B}}$ give
rise to acceptable inductive-inductive definitions. It is well known,
for example, that the theory becomes inconsistent if $A$ or $B$ occur
in negative positions in $\Phi_{\mathrm{A}}$ or $\Phi_{\mathrm{B}}$
respectively. Thus, we restrict our attention to a class of strictly
positive functors.

%todo: non-ind, A-ind, B-ind

\subsection{The axiomatisation}
\label{sec:formal-axiomatisation}

We now give the formal rules for an inductive-inductive definition of
$A : \Set$, $B : A \to \Set$. These consists of a set of rules for the
universe $\SPA$ of descriptions of the set $A$ and its decoding
function $\ArgA$, a set of rules for the universe $\SPB$ and its
decoding function $\ArgB$, and formation and introduction rules for
each pair of codes $\gammaA : \SPA$, $\gammaB : \SPB(\gammaA)$.

%elimination rules (with only the type of \BOX_F)

\subsubsection{The universe $\SPA$ of descriptions of $A$}
\label{sec:SPA}
  
We introduce the universe of codes for the index set with the
formation rule
\[
\infer{\SPA(\Aref) : \TYPE}{\Aref : \Set}
\]
and define (by abuse of notation) $\SPA \coloneqq \SPA(\zero)$. Thus
codes in $\SPA$ do not refer to any elements to start with.

The introduction rules for $\SPA$ reflects the informal discussion in
Section~\ref{sec:dissect-ind}. The rules are as follows (we suppress
the global premise $\Aref : \Set$):
\[
\infer{\nilA : \SPA(\Aref)}{}
\]
%
The code $\nilA$ represents a trivial constructor $c : \one \to A$ (a base case).
%
\[
\infer{\nonindA(K, \gamma) : \SPA(\Aref)}{K : \Set & \quad \gamma : K \to \SPA(\Aref)}
\]
%
The code $\nonindA(K, \gamma)$ represents a non-inductive argument $x
: K$, with the rest of the arguments given by $\gamma(x)$.
%
\[
\infer{\AindA(K, \gamma) : \SPA(\Aref)}{K : \Set & \quad \gamma : \SPA(\Aref + K)}
\]
%
The code $\AindA(K, \gamma)$ represents an inductive argument with
type $K \to A$, with the rest of the arguments given by
$\gamma$. Notice that $\gamma : \SPA(\Aref + K)$, so that the
remaining arguments can refer to more elements in $A$ (namely those
introduced by the inductive argument).
%
\[
\infer{\BindA(K, \hindex, \gamma) : \SPA(\Aref)}{K : \Set & \quad \hindex : K \to \Aref & \quad \gamma : \SPA(\Aref)}
\]
%
Finally, the code $\BindA(K, \hindex, \gamma)$ represents an inductive
argument with type $(x : K) \to B(i(x))$, where the index $i(x)$ is
determined by $\hindex$, and the rest of the arguments are given by
$\gamma$.
%

\begin{example}
  The constructor $\consCtxtbare : ((\Gamma : \Ctxt) \times \Ty(\Gamma))
  \to \Ctxt$ is represented by the code
\[
\gamma_{\consCtxtbare} = \AindA(\one, \BindA(\one, \widehat{\Gamma}, \nilA)) \enspace ,
\]
where $\widehat{\Gamma} = \lambda (\oneelt : \one)\,.\,\inr(\oneelt)$ is the representation of $\Gamma$ in $\Aref = \zero + \one$.
\blackqed
\end{example}

We now define the decoding function $\ArgA$, which maps a code to the
domain of the constructor it represents. In addition to a set $\Aref$
and a code $\gamma : \SPA(\Aref)$, $\ArgA$ will take a set $X$ and a
family $Y : X \to \Set$ as arguments to use as $A$ and $B$ in the
inductive arguments. These will later be instantiated recursively. We
also require a function $\repA : \Aref \to X$ which we think of as
mapping a ``referable'' element to the element it represents in
$X$. All in all, $\ArgA$ has the following formation rule:
%
\[
\infer{\ArgA(\Aref, \gamma. X, Y, \repA) : \Set}{\Aref : \Set
                                         & \quad \gamma : \SPA(\Aref)
                                         & \quad X : \Set
                                         & \quad Y : X \to \Set
                                         & \quad \repA : \Aref \to X}
\]
%
Notice that if $\gamma : \SPA$, i.e.\ if $\Aref = \zero$, then we can
choose $\repA = {} \magic{X} : \zero \to X$ (indeed, extensionally,
this is the only choice), so that we can define
\[
\ArgA : \SPA \to (X : \Set) \to (Y : X \to \Set) \Set
\]
by $\ArgA(\gamma, X, Y) = \ArgA(\zero, \gamma, X, Y, \magic{X})$.

The definition of $\ArgA$ follows the informal description of what the
different codes represent above:

%todo: formatting %todo mark changing arguments
\begin{align*} 
  \ArgA(\Aref, \nilA, X, Y, \repA) &= \one \\
  \ArgA(\Aref, \nonindA(K, \gamma), X, Y, \repA) &= (x : K) \times \ArgA(\Aref, \gamma(x), X, Y, \repA) \\
  \ArgA(\Aref, \AindA(K, \gamma), X, Y, \repA) &=  (j : K \to X) \times \ArgA(\Aref + K, \gamma, X, Y, [\repA, j]) \\
  \ArgA(\Aref, \BindA(K, \hindex, \gamma), X, Y, \repA) &=  ((x : K) \to Y((\repA \circ \hindex)(x))) \times \ArgA(\Aref, \gamma, X, Y, \repA) \\
\end{align*}

%todo: discuss Aind and Bind

\begin{example}
  Recall the code $\gamma_{\consCtxtbare} = \AindA(\one, \BindA(\one,
  \lambda (\oneelt : \one)\,.\,\inr(\oneelt), \nilA))$ for the
  constructor $\consCtxtbare : ((\Gamma : \Ctxt) \times \Ty(\Gamma)) \to
  \Ctxt$. We have 
\[
\ArgA(\gamma_{\consCtxtbare}, \Ctxt, \Ty) = (\Gamma : \one \to \Ctxt)
\times (\one \to \Ty(\Gamma(\oneelt))) \times \one
\]
which, thanks to the $\eta$-rules for $\one$, is isomorphic to the
domain of $\consCtxtbare$. 
\blackqed
\end{example}

\subsubsection{Towards descriptions of $B$} %todo: rename
\label{sec:towards-SPB}

As we have seen in Example \ref{ex:ctxt-type}, it is important that
the constructor $\introB$ for the second set $B : A \to \Set$ can
refer to the constructor $\introA$ for the first set $A$. This means
that inductive arguments might be of type $B(\introA(\overline{a}))$
for some $\overline{a} : \ArgA(\gammaA, A, B)$ or even
$B(\introA(\ldots\introA\ldots(\overline{a})))$ for some $\overline{a}
: \ArgA(\gammaA, \ldots \ArgA(\gammaA, A, B)\ldots, B')$. Thus, we
need to be able to represent such indices in the descriptions of the
constructor $\introB$.

First, it is no longer enough to only keep track of the referable
elements $\Aref$ of $A$ -- we need to be able to refer to elements of
$B$ as well, since they could be used as arguments to $\introA$. We
will represent the elements of $B$ we can refer to by a set $\Bref$,
together with functions $\repIndex : \Bref \to A$ and $\repB : (x :
\Bref) \to B(\repIndex(x))$ ; the function $\repIndex$ gives the index
of the represented element, and $\repB$ the actual element.

%todo: consistent use of X, Y / A , B

We want to represent elements in $\ArgA(\gammaA, A, B)$. We claim that
the elements in $\ArgA(\gammaA, \Aref + \Bref, [\lambda x\,.\,\zero,
\lambda x\,.\,\one])$ are suitable for this purpose. To see this,
first observe that we can define a functions
\[
f : \Aref + \Bref \to A
\]
%
%and
%
\[
g : (x : \Aref + \Bref) \to
    [\lambda x\,.\,\zero, \lambda x\,.\,\one](x)
      \to B(f(x))
\]
%
by $f = [\repA, \repIndex]$ and $g = [\lambda x\,.\,\magic{B \circ
  \repA}, \lambda x \oneelt\,.\, \repB(x)]$. Then, we can lift these
functions to a function
%
\[
\ArgA(\gammaA, f, g) : \ArgA(\gammaA, \Aref +\Bref, [\lambda x\,.\,\zero, \lambda x\,.\,\one]) \to
                       \ArgA(\gammaA, A, B)
\]
%
by observing that $\ArgA(\gammaA)$ is functorial:

\begin{lemma} %todo: reword. Do we want to avoid cat. theory?
  For each $\gamma : \SPA$, $\ArgA(\gamma)$ extends to a functor from
  families of sets to sets, i.e.\ given $f : A \to A'$ and $g : (x :
  A) \to B(x) \to B'(f(x))$, one can define $\ArgA(\gamma, f, g) :
  \ArgA(\gamma, A, B) \to \ArgA(\gamma, A', B')$ such that identities
  and compositions are preserved.
\end{lemma}
\begin{proof} %todo

  This is straightforward in extensional type theory. In intensional
  type theory, we have to be more careful. The function $\ArgA(\gamma,
  f, g)$ is defined by induction over $\gamma$ and $\Aref$: let $A$,
  $B$, $\repA : \Aref \to A$, $A'$, $B'$, $\repA' : \Aref \to A'$ and
  $f : A \to A'$, $g : (x : A) \to B(x) \to B'(f(x))$ be given. In
  addition, assume that $f(\repA(x)) = \repA'(x)$ for all $x :
  \Aref$. It is enough for our purposes to state this in a
  ``specialised Leibniz form''; we require a term
  \[
  p : (x : \Aref) \to B'(f(\repA(x))) \to B'(\repA'(x)) \enspace .
  \]
  % 
  Thus we define
  % 
  \[
  \ArgA(\gamma, f, g, p) : %(p : (x : \Aref) \to B'(f(\repA(x))) \to B'(\repA'(x))) \to
  \ArgA(\Aref, \gamma, A, B, \repA)
  \to \ArgA(\Aref, \gamma, A', B', \repA')
  \]
  % 
  by induction over $\gamma$:
  \begin{align*}
    \ArgA(\nilA, f, g, p, \oneelt) &= \oneelt\\
    \ArgA(\nonindA(K, \gamma), f, g, p, \pair{k}{y}) &= \pair{k}{\ArgA(\gamma(k), f, g, p, y)}\\
    \ArgA(\AindA(K, \gamma), f, g, p, \pair{j}{y}) &= \pair{f \circ j}{\ArgA(\gamma, f, g, [p, \lambda x\,.\,\id], y)}\\
    \ArgA(\BindA(K, \hindex, \gamma), f, g, p, \pair{j}{y}) &= \pair{\lambda k\,.\,p(h(k), g(\repA(h(k)), j(k)))}{\ArgA(\gamma, f, g, p, y)}
  \end{align*}
  
  Finally, we can define $\ArgA(\gamma, f, g) : \ArgA(\gamma, A, B) \to
  \ArgA(\gamma, A', B')$ by
  \[
  \ArgA(\gamma, f, g) \coloneqq \ArgA(\gamma,f, g, \magic{B'(f(\repA(x))) \to B'(\repA'(x))} \enspace .
  \]
  % 
  (Pointwise) preservation of identities and composition is easily checked.
\end{proof}

Recall that we want to use the lemma to represent elements in
$\ArgA(\gammaA. A, B)$ by elements in $\ArgA(\gammaA, \Aref +\Bref,
[\lambda x\,.\,\zero, \lambda x\,.\,\one])$. We can actually do one
better, and represent arbitrarily terms built from elements in $A$ and
$B$ with the use of a constructor $\introA : \ArgA(\gammaA, A, B) \to
A$. For this, define $\argn{n}$ with formation rule
%
\[
\infer{\argn{n}(\gammaA, \Aref, \Bref) : \Set}{n \in \Nat & \quad \gammaA : \SPA & \quad \Aref, \Bref : \Set}
\]
%
by letting
\begin{align*}
\argn{0}(\gammaA, \Aref, \Bref) &= \Aref \\
\argn{n + 1}(\gammaA, \Aref, \Bref) &= \ArgA(\gammaA, \bigplus_{k = 0}^n \argn{k}(\gammaA, \Aref, \Bref) + \Bref,  [\lambda x\,.\,\zero, \lambda x\,.\,\one])
\end{align*}

%todo: motivate \bigplus

Given a constructor $\introA : \ArgA(\gammaA, A, B) \to A$, all
elements in $\argn{n}(\gammaA,\Aref, \Bref)$ represents elements in
$A$, given that the elements of $\Aref$ and $\Bref$ represents elements
of $A$ and $B$ respectively (i.e.\ we have $\repA : \Aref \to A$,
$\repIndex : \Bref \to A$ and $\repB : (x : \Bref) \to
B(\repIndex(x))$):
%
\[
\infer{\repn{n}(\gammaA, \introA, \repA, \repIndex, \repB) : \argn{n}(\gammaA, \Aref, \Bref) \to A}
      {n : \Nat & \quad
       \gammaA : \SPA & \quad
       \introA : \ArgA(\gammaA, A, B) \to A & \quad
              %{\deduce{B : A \to \Set}
              %        {A : \Set}} & \quad
       \deduce{\repB : (x : \Bref) \to B(\repIndex(x))}
              {\deduce{\repIndex : \Bref \to A}
                      {\repA : \Aref \to A}}}
\]
%
\begin{align*}
\repn{0}(\ldots) &= \repA \\
\repn{n + 1}(\ldots) &= \introA \circ \ArgA(\gammaA, [\repn{0}(\ldots), \ldots, \repn{n}(\ldots)], [\lambda x\,.\,\magic{}, \lambda x \oneelt\,.\,\repB(x)])
\end{align*}

%todo: say something about the use of complete induction?

Define the set of terms ``built from $\introA$, $\Aref$ and  $\Bref$'' as
%
\[
\Aterm(\gammaA, \Aref, \Bref) = (n : \Nat) \times \argn{n}(\gammaA, \Aref, \Bref) \enspace .
\]
%
Each term can be mapped to the element in $A$ it denotes:
%
\[
\infer{\repbar(\gammaA, \introA, \repA, \repIndex, \repB) : \Aterm(\gammaA, \Aref, \Bref) \to A}
      {\gammaA : \SPA & \quad
       \introA : \ArgA(\gammaA, A, B) \to A & \quad
              %{\deduce{B : A \to \Set}
              %        {A : \Set}} & \quad
       \deduce{\repB : (x : \Bref) \to B(\repIndex(x))}
              {\deduce{\repIndex : \Bref \to A}
                      {\repA : \Aref \to A}}}
\]
%
$\repbar(\ldots, \pair{n}{x}) = \repn{n}(\ldots, x)$


\subsubsection{The universe $\SPB$ of descriptions of $B$}
\label{sec:SPB}

We now introduce the universe $\SPB : \SPA \to \TYPE$ of descriptions
for $B$. It has formation rule
%
\[
\infer{\SPB(\Aref, \Bref, \gammaA) : \TYPE}{\Aref, \Bref : \Set & \quad \gammaA : \SPA}
\]
%
Again, we abuse notation and define $\SPB : \SPA \to \TYPE$ by
$\SPB(\gammaA) \coloneqq \SPA(\zero, \zero, \gammaA)$.

The introduction rules for $\SPB$ are similar to the ones for $\SPA$,
except we now need to specify an index for the codomain of the
constructor, and indices for arguments inductive in $B$ can be
arbitrary terms built up from $\introA$ and elements we can refer to.

\[
\infer{\nilB(a) : \SPB(\Aref, \Bref, \gammaA)}{a : \Aterm(\gammaA, \Aref, \Bref)}
\]
%
%The code $\nilA$ represents a trivial constructor $c : \one \to A$ (a base case).
%
\[
\infer{\nonindB(K, \gamma) : \SPB(\Aref, \Bref, \gammaA))}{K : \Set & \quad \gamma : K \to \SPB(\Aref, \Bref, \gammaA)}
\]
%
%The code $\nonindA(K, \gamma)$ represents a non-inductive argument $x: K$, with the rest of the arguments given by $\gamma(x)$.
%
\[
\infer{\AindB(K, \gamma) : \SPB(\Aref, \Bref, \gammaA)}{K : \Set & \quad \gamma : \SPB(\Aref + K, \Bref, \gammaA)}
\]
%
%The code $\AindA(K, \gamma)$ represents an inductive argument with
%type $K \to A$, with the rest of the arguments given by
%$\gamma$. Notice that $\gamma : \SPA(\Aref + K)$, so that the
%remaining arguments can refer to more elements in $A$ (namely those
%introduced by the inductive argument).
%
\[
\infer{\BindB(K, \hindex, \gamma) : \SPB(\Aref, \Bref, \gammaA)}{K : \Set & \quad \hindex : K \to \Aterm(\Aref, \Bref, \gammaA) & \quad \gamma : \SPB(\Aref, \Bref + K, \gammaA)}
\]
%
%Finally, the code $\BindA(K, \hindex, \gamma)$ represents an inductive
%argument with type $(x : K) \to B(i(x))$, where the index $i(x)$ is
%determined by $\hindex$, and the rest of the arguments are given by
%$\gamma$.
%
\begin{example}
  The constructor
  % 
  \[
  \Pi : \big((\Gamma : \Ctxt) \times (\sigma : \Ty(\Gamma)) \times \Ty(\consCtxt{\Gamma}{\sigma})\big)\to \Ty(\Gamma)
  \]
  % 
  is represented by the code
  % 
  \[
  \gamma_{\Pi} = \AindB(\one,
                  \BindB(\one, \lambda \oneelt\,.\,\widehat{\Gamma}, 
                    \BindB(\one, \lambda \oneelt\,.\,\widehat{\text{in}\pair{\Gamma}{\sigma}},
                      \nilB(\widehat{\Gamma}))))
  \]
  %
  where $\widehat{\Gamma} = \pair{0}{\inr(\oneelt)}$ is the element
  representing the first argument $\Gamma : \Ctxt$ and
  $\widehat{\text{in}\pair{\Gamma}{\sigma}} =
  \pair{1}{\pair{\twott}{\pair{(\lambda
        \oneelt\,.\,\inr(\inr(\oneelt)))}{\pair{(\lambda
          \oneelt\,.\,\oneelt)}{\oneelt}}}}$ is the element
  representing $\consCtxt{\Gamma}{\sigma}$ (the number $1$ corresponds
  to the fact that we are using one nesting of constructors).
\blackqed
\end{example}

The definition of $\ArgB$ should now not come as a surprise. First, we have a formation rule:
%
\[
\mathclap{%
\infer{\ArgB(\Aref, \Bref, \gammaA, A, B, \introA, \repA, \repIndex, \repB, \gamma) : \Set}
                                        {\deduce{\Aref, \Bref : \Set}{\gammaA : \SPA}
                                         & \quad \deduce{\introA : \ArgA(\gammaA, A, B) \to A}
                                                        {\deduce{B : A \to \Set}{A : \Set}}
                                         & \quad \deduce{\repB: (x : \Bref) \to B(\repIndex(x))}
                                                        {\deduce{\repIndex : \Bref \to A}
                                                                {\repA : \Aref \to X}}
                                         & \quad \gamma : \SPB(\Aref, \Bref, \gammaA)}
}
\]
%
Once again, the definition is simpler for codes in $\SPB(\gammaA)$:
\[
\ArgB(\gammaA, A, B, \introA, \gamma) \coloneqq
   \ArgB(\zero, \zero, \gammaA, A, B, \introA, \magic{A}, \magic{A}, \magic{B \circ \magic{A}}, \gamma)
\]
%
We define:
%todo: formatting
\begin{align*} 
&  \ArgB(\Aref, \Bref, \gammaA, A, B, \introA, \repA, \repIndex, \repB, \nilB(a)) = \one \\
&  \ArgB(\Aref, \Bref, \gammaA, A, B, \introA, \repA, \repIndex, \repB, \nonindB(K, \gamma)) \\  
&\quad= (x : K) \times \ArgB(\Aref, \Bref, \gammaA, A, B, \introA, \repA, \repIndex, \repB, \gamma(x)) \\
&  \ArgB(\Aref, \Bref, \gammaA, A, B, \introA, \repA, \repIndex, \repB, \AindB(K, \gamma)) \\
&\quad=  (j : K \to A) \times \ArgB(\Aref + K, \Bref, \gammaA, A, B, \introA,  [\repA, j], \repIndex, \repB, \gamma) \\
&  \ArgB(\Aref, \Bref, \gammaA, A, B, \introA, \repA, \repIndex, \repB, \BindB(K, \hindex, \gamma)) \\
&\quad= (j : (x : K) \to B((\repbar(\gammaA, \introA, \repA, \repIndex, \repB) \circ \hindex)(x))) \times {} \\ 
& \qquad\quad     \ArgB(\Aref, \Bref + K, \gammaA, A, B, \introA, \repA, [\repIndex, \repbar(\ldots) \circ \hindex], [\repB, j], \gamma)
\end{align*}

Finally, we need the function $\IndexB(\ldots) : \ArgB(\gammaA, \gammaB, A, B,
\introA) \to A$ which to each $b : \ArgB(\gammaA, \gammaB, A, B,
\introA)$ assigns an index $a : A$ such that the element constructed
from $b$ is in $B(a)$.
%
\[
\mathclap{%
\infer{\IndexB(\Aref, \Bref, \gammaA, A, B, \introA, \repA, \repIndex, \repB, \gamma) : \ArgB(\ldots) \to A}
                                        {\deduce{\Aref, \Bref : \Set}{\gammaA : \SPA}
                                         & \quad \deduce{\introA : \ArgA(\gammaA, A, B) \to A}
                                                        {\deduce{B : A \to \Set}{A : \Set}}
                                         & \quad \deduce{\repB: (x : \Bref) \to B(\repIndex(x))}
                                                        {\deduce{\repIndex : \Bref \to A}
                                                                {\repA : \Aref \to X}}
                                         & \quad \gamma : \SPB(\Aref, \Bref, \gammaA)}
}
\]
%
As is standard for us at this point, we define $\IndexB :
\ArgB(\gammaA, A, B, \introA, \gammaB) \to A$ by
%
\[
\IndexB(\gammaA, A,B, \introA, \gammaB) \coloneqq
   \IndexB(\zero, \zero, \gammaA, A, B, \introA, \magic{A}, \magic{A}, \magic{B \circ \magic{A}}, \gammaB)
\]
%
For the equations, we will suppress all inferable arguments:
%
\begin{align*} 
\IndexB(\ldots, \nilB(a), \oneelt) &= \repbar(\ldots, a) \\
\IndexB(\ldots, \nonindB(K, \gamma), \pair{k}{y}) &= \IndexB(\ldots, \gamma(k), y) \\  
\IndexB(\ldots, \AindB(K, \gamma), \pair{j}{y})  &= \IndexB(\ldots, \gamma, y) \\
\IndexB(\ldots, \BindB(K, \hindex, \gamma), \pair{j}{y})  &= \IndexB(\ldots, \gamma, y)
\end{align*}

\begin{example}
  $\ArgB$ and $\IndexB$ of $\gamma_{\Pi}$.
\blackqed
\end{example}

\subsubsection{Formation and introduction rules}
\label{sec:intro-elim}

We are now ready to give the formation and introduction rules for $A$
and $B$. They all have the common premises $\gammaA : \SPA$, $\gammaB
: \SPB(\gammaA)$, which will be omitted.

Formation rules:
%
\[
A_{\gammaAB} : \Set \qquad\quad
B_{\gammaAB} : A_{\gammaAB} \to \Set
\]
%
Introduction rule for $A_{\gammaAB}$:
%
\[ %todo: check PML convention on variable names (x vs a)
\infer{\intro{A_{\gammaAB}}(x) : A_{\gammaAB}}{x : \ArgA(\gammaA, A_{\gammaAB}, B_{\gammaAB})}
\]
%
Introduction rule for $B_{\gammaAB}$:
%
\[
\infer{\intro{B_{\gammaAB}}(x) : B_{\gammaAB}(\IndexB(\gammaA, A_{\gammaAB}, B_{\gammaAB}, \intro{A_{\gammaAB}}, \gammaB, x))}{x : \ArgB(\gammaA, A_{\gammaAB}, B_{\gammaAB}, \intro{A_{\gammaAB}}, \gammaB)}
\]
%

Elimination rules can also be
formulated~\cite{nordvallforsbergAltenkirchMorrisSetzer2011catsemindind}. Here,
we just give the elimination rules for the data type of sorted lists
(Example~\ref{ex:sorted-list}) as an example, and show how one can use them to
define a function which inserts a number into a sorted list.

\begin{example}
   %todo: or simply Ctxt + types?
\blackqed
\end{example}

\subsection{The examples revisited}
\label{sec:examples-revisited}

%\blackqed

We show how to find \gammaAB{} for some well-known sets, including the
examples in Section~\ref{sec:examples}.

%natural numbers

%vectors etc

%ctxts and types


\section{A set-theoretic model}
\label{sec:model}

%todo: rewrite from CSL article

Even though $\SPA$ and $\SPB$ themselves are straightforward (large)
inductive definitions, this axiomatisation does not reduce
inductive-inductive definitions to indexed inductive definitions,
since the formation and introduction rules are not instances of
ordinary indexed inductive definitions.  (However, we do believe that
the theory of inductive-inductive definitions \emph{can} be reduced to
the theory of indexed inductive definitions with a bit of more
work, %and plan to publish an article about this in the future.
see Section~\ref{sec:indexed-induction}.) To make sure that our theory
is consistent, it is thus neccessary to construct a model.


We will develop a model in ZFC set theory, extended by two
inaccessible cardinals in order to interpret $\Set$ and $\TYPE$.  Our
model will be a simpler version of the models developed in
\cite{dybjersetzer1999finax,dybjersetzer2006IIR}. %Hence the proof theoretical
%strength required is, as expected significantly lower but still too
%strong. 
See Aczel \cite{aczel1999typesandsets} for a more detailed treatment
of interpreting type theory in set theory.



\subsection{Preliminaries}

We will be working informally in ZFC extended with the existence of two
strongly inaccessible cardinals $\cardz < \cardi$,
and will be using standard set theoretic constructions, e.g.
\begin{align*}
\langle a, b\rangle &\coloneqq \{ \{ a \}, \{ a, b \}\}, \\
\lambda x \in a.b(x) &\coloneqq \{\langle x, b(x)\rangle | x \in a \} \\
\Pi_{x \in a}b(x) &\coloneqq \{ f : a \to \bigcup_{x \in a}b(x)\ |\ \forall x \in a.f(x) \in b(x)\}, \\
\Sigma_{x \in a}b(x) &\coloneqq \{ \langle c , d\rangle\ |\ c \in a \land d
\in b(c) \}, \\
0 &\coloneqq \emptyset, 1 \coloneqq \{0\}, 2 \coloneqq \{0, 1\}, \ldots, \\
\Nat &\coloneqq \text{smallest set containing $0$ and closed under successor} \\
a_0 + \ldots + a_n &\coloneqq \Sigma_{i \in \{0, \ldots, n\}}a_i
\end{align*}
and the cumulative hierarchy $V_{\alpha} \coloneqq
\displaystyle\bigcup_{\beta < \alpha} \mathcal{P}(V_{\beta})$. Whenever we
introduce sets $A^{\alpha}$ indexed by ordinals $\alpha$, let \[A^{< \alpha}
\coloneqq \displaystyle\bigcup_{\beta < \alpha}A^{\beta}.\]

For every expression $A$ of our type theory, we will give an interpretation
$\sem{A}_\rho$, regardless if $A : \TYPE$ or $A : B$ or not. Interpretations
might however be undefined, written $\sem{A}_\rho\uparrow$. If $\sem{A}_\rho$
is defined, we write $\sem{A}_\rho\downarrow$. We write $A \simeq B$ for
partial equality, i.e.\ $A \simeq B$ if and only if $A\downarrow
\Leftrightarrow B\downarrow$ and if $A\downarrow$, then $A = B$. We write $A
\colonsimeq B$ if we define $A$ such that $A \simeq B$.

Open terms will be interpreted relative to an environment $\rho$, i.e.\ a
function mapping variables to terms. Write \rhoextend{x}{a} for the
environment $\rho$ extended with $x \mapsto a$, i.e.\ $\rhoextend{x}{a}(y) = a$
if $y = x$ and $\rho(y)$ otherwise. The interpretation $\sem{t}_{\rho}$ of closed
terms $t$ will not depend on the environment, and we omit the subscript $\rho$.

\subsection{Interpretation of Expressions}
\noindent 
The interpretation of the logical framework is as in
\cite{dybjersetzer1999finax}:
\begin{gather*}
\sem{\Set} \colonsimeq V_{\cardz} \qquad \sem{\TYPE} \colonsimeq V_{\cardi} \\
\sem{(x : A) \to B}_\rho \colonsimeq \Pi_{y \in
  \sem{A}_\rho}\sem{B}_{\rhoextend{y}{x}} \qquad
\sem{\lambda x:A.e}_\rho \colonsimeq \lambda y \in
\sem{A}_\rho.\sem{e}_{\rhoextend{y}{x}} \\
\sem{(x : A) \times B}_\rho \colonsimeq \Sigma_{y \in
  \sem{A}_\rho}\sem{B}_{\rhoextend{y}{x}} \qquad
\sem{\langle a, b \rangle}_\rho \colonsimeq \langle \sem{a}_\rho ,
\sem{b}_\rho\rangle \\
\sem{\zero} \colonsimeq 0 \qquad \sem{\one} \colonsimeq 1 \qquad
\sem{\two} \colonsimeq 2 \qquad \sem{\oneelt} \colonsimeq 0 \qquad \sem{\twott} \colonsimeq 0 \qquad
\sem{\twoff} \colonsimeq 1\\
\sem{\IF~x~\THEN~a~\ELSE~b}_\rho \colonsimeq
\begin{cases}
\sem{a}_\rho & \text{if $\sem{x}_\rho = 0$} \\
\sem{b}_\rho & \text{if $\sem{x}_\rho = 1$} \\
\text{undefined} & \text{otherwise} \\
\end{cases} \\
\sem{!_A}_\rho \colonsimeq \text{unique inclusion $\iota_{\sem{A}} : \emptyset \to \sem{A}_\rho$ }
\end{gather*}

To interpret terms containing $\SPA$, $\SPB$, $\ArgA$, $\ArgB$, $\IndexB$, 
$\nilA$, $\nonindA$, $\AindA$, $\BindA$, we first define $\sem{\SPA}$,
$\sem{\SPB}$, $\sem{\ArgA}$, $\sem{\nilA}$, $\sem{\nonindA}$, \ldots and interpret
\begin{align*}
\sem{\SPA(\Aref)}_\rho &\coloneqq \sem{\SPA}(\sem{\Aref}_\rho) \\
\vdots \\
\sem{\ArgA(\Aref, \gamma, A, B, \repA)}_\rho &\coloneqq
\sem{\ArgA}(\sem{\Aref}_\rho, \sem{\gamma}_\rho, \sem{A}_\rho, \sem{B}_\rho, \sem{\repA}_\rho) \\
\vdots \\
\sem{\nonindA(K, \gamma)}_\rho &\coloneqq \sem{\nonindA}(\sem{K}_\rho,
\sem{\gamma}_\rho) \\
\vdots & \text{\quad etc.}
\end{align*}
In all future definitions, if we are currently defining $\sem{F}$ where $F : D
\to E$, say, let $F(d)\uparrow$ if $d \notin \sem{D}$.

$\sem{\SPA}(\Aref)$ is defined as the least set such that
\begin{align*}
\sem{\SPA}(\Aref) &= 1 + \sum_{K \in \sem{\Set}}(K \to \sem{\SPA}(\Aref))
  + \sum_{K \in \sem{\Set}}\sem{\SPA}(\Aref + K) \\
 & {} + \sum_{K \in \sem{\Set}}\sum_{h : K \to \Aref}\sem{\SPA}(\Aref).
\end{align*}
 By the inaccessibility of $\cardi$, there is a
regular cardinal ${\kappa < \cardi}$ such that for all $K \in \sem{\Set}$, we
have that the cardinality of $K$, $\Aref$, $\Aref + K$ and $(K \to \Aref)$ is less than
$\kappa$. If we now iterate an appropriate operator $\kappa$ times, we get our
solution, which must be an element of $\sem{\TYPE} = V_{\cardi}$ by the
inaccessibility of $\cardi$.
\begin{align*}
 \sem{\nilA} \colonsimeq \langle 0, 0\rangle \quad
 \sem{\BindA(K, h, \gamma)} \colonsimeq \langle 3 , \langle K, \langle h, \gamma \rangle\rangle\rangle
   \\
  \sem{\nonindA(K, \gamma)} \coloneqq \langle 1 , \langle K, \gamma
  \rangle\rangle \quad \sem{\AindA(K, \gamma)} \colonsimeq \langle 2 , \langle K, \gamma \rangle\rangle
\end{align*}
$\sem{\SPB}$ and its constructors are defined
analogously. $\sem{\ArgA}$, $\sem{\ArgB}$ and $\sem{\IndexB}$ are defined according to their equations,
e.g.\ 
\begin{align*}
  \sem{\ArgA}(\Aref, \sem{\nilA}, A, B, \repA) &\colonsimeq 1 \\
\sem{\ArgA}(\Aref, \sem{\nonindA}(K, \gamma), A. B. \repA) &\colonsimeq
  \sum_{k \in K} \sem{\ArgA}(\Aref, \gamma(k), A, B, \repA) \\
\sem{\ArgA}(\Aref, \sem{\AindA}(K, \gamma), A, B, \repA) &\colonsimeq
  \sum_{j : K \to A}\sem{\ArgA}(\Aref + K, \gamma, A, B, [\repA, j]) \\
\sem{\ArgA}(\Aref, \sem{\BindA}(K, h, \gamma), A, B, \repA) &\colonsimeq
  \sum_{\mathclap{j \in \Pi_{k \in K} B(\repA (h(k)))}}\sem{\ArgA}(\Aref, \gamma, A, B, \repA).
\end{align*}
%
% todo: rework to new axiomatisation
%

Finally, we have to interpret $A_{\gammaAB}$, $B_{\gammaAB}$,
$\intro{A_{\gammaAB}}$ and $\intro{B_{\gammaAB}}$.  The high-level
idea is to iterate $\ArgA$ until a fixed point is reached, then apply
$\ArgB$ once, and repeat. This is necessary since $\ArgB$ expects an
argument $\introA : \ArgA(\gammaA, A, B) \to A$, which can be chosen
to be the identity if $A$ is a fixed point of $\ArgA(\gammaA, A, B)$
(with $B$ fixed). In more detail, let
\begin{gather*}
\sem{A_{\gammaAB}} \colonsimeq A^{\cardz} \enspace\! , \quad\!\!
\sem{B_{\gammaAB}}(a) \colonsimeq B^{\cardz}(a)\enspace\! , \quad\!\!
\sem{\intro{A_{\gammaAB}}}(a) \colonsimeq a \enspace\! ,\quad\!\!
\sem{\intro{B_{\gammaAB}}}(b) \colonsimeq b \enspace\! ,\\
\end{gather*} \vskip -0.5cm \noindent 
where $A^{\alpha}$ and $B^{\alpha}$ are
simultaneously defined by recursion on $\alpha$ as
\begin{align*}
  A^{\alpha} & \coloneqq  \text{least fixed point of }
       \lambda X\,.\, \sem{\ArgA}(\gamma_A, A^{< \alpha} \cup X, [B^{< \alpha}, \lambda x\,.\,\emptyset]) \enspace ,\\
  B^{\alpha}(a) & \coloneqq \{b\ |\ b \in \sem{\ArgB}(\gamma_A, A^{< \alpha} \cup A^{\alpha},
  [B^{< \alpha}, \lambda x\,.\,\emptyset], \id, \gamma_B) \\
  & \quad \land \sem{\IndexB}(\gamma_A,
  A^{< \alpha} \cup A^{\alpha}, [B^{< \alpha}, \lambda x\,.\,\emptyset], \id, \gamma_B, b) = a \}\enspace \hspace{-2.47968pt} .
\end{align*}

The (graph of the) eliminators can then be built up in the same
stages.

Having interpreted all terms, we finally interpret contexts as sets of environments:
\begin{align*}
  \sem{\emptyset} \colonsimeq \emptyset && \sem{\Gamma, x : A} \colonsimeq \{
  \rhoextend{x}{a} \ |\ \rho \in \sem{\Gamma} \land a \in \sem{A}_\rho\}.
\end{align*}

\subsection{Soundness of the Rules}
\begin{theorem}[Soundness] \mbox{}
\label{thm:soundness}
  \begin{enumerate}[(i)]
  \item If ${} \tstile \Gamma \text{context}$, then
    $\sem{\Gamma}\downarrow$. 
  \item If $\Gamma \tstile A : E$, then $\sem{\Gamma}\downarrow$, and
    for all $\rho \in \sem{\Gamma}$, ${\sem{A}_\rho \in \sem{E}_\rho}$, and also
    $\sem{E}_\rho \in \sem{\TYPE}$ if $E \not\equiv \TYPE$.
  \item If $\Gamma \tstile A = B : E$, then $\sem{\Gamma}\downarrow$, and
    for all $\rho \in \sem{\Gamma}$, $\sem{A}_\rho = \sem{B}_\rho$,
    $\sem{A}_\rho \in \sem{E}_\rho$ and also $\sem{E}_\rho \in \sem{\TYPE}$ if
    $E \not\equiv \TYPE$. 
  \item $\nvdash a : \zero$. \qed
  \end{enumerate}
\end{theorem}
The proof of the soundness theorem is rather
  straight\-forward. For the verification that $\sem{A_{\gamma_A, \gamma_B}}
  \in \sem{\Set}$ and $\sem{B_{\gamma_A, \gamma_B}} : \sem{A_{\gamma_A,
      \gamma_B}} \to \sem{\Set}$, one first verifies that $\sem{\ArgA}$,
  $\sem{\ArgB}$, $\sem{\IndexB}$ are monotone in the following sense:
  %
  \begin{lemma}
  \label{thm:Arg-monotone}
  For all $\gammaA \in \sem{\SPA}$ and $\gammaB \in \sem{\SPB}(\gammaA)$:
  \begin{enumerate}[(i)]
  \item If $A \subseteq A'$ and $B(x) \subseteq B'(x)$ then
    $\sem{\ArgA}(\gammaA, A, B) \subseteq \sem{\ArgA}(\gammaA, A', B')$.
  \item If in addition $\introA(x) = \introA'(x)$ for all $x \in \ArgA(\gammaA, A, B)$, then
    \[
    \sem{\ArgB}(\gammaA, A, B, \introA, \gammaB) \subseteq \sem{\ArgA}(\gammaA, A', B', \introA', \gammaB)
    \]
    and
    \[    
    \sem{\IndexB}(\gammaA, A, B, \introA, \gammaB, x) = \sem{\IndexB}(\gammaA, A', B', \introA', \gammaB, x)
    \]
    for all $x \in \sem{\ArgB}(\gammaA, A, B, \introA, \gammaB)$. \qed
  \end{enumerate}
  \end{lemma}

One can then prove some useful facts about $A^{\alpha}$ and
  $B^{\alpha}$: % by induction on $\alpha$:
\begin{lemma} \mbox{}
\label{thm:A-in-Set}
  \begin{enumerate}[(i)]
  \item For $\alpha < \cardz$, $A^{\alpha} \in \sem{\Set}$ and $B^{\alpha} :
    A^{\alpha} \to \sem{\Set}$.
  \item For $\alpha < \beta$, $A^{\alpha} \subseteq A^{\beta}$ and  $B^{\alpha}(a) \subseteq
    B^{\beta}(a)$ for all $a \in A^{\alpha}$. 
  \item There is $\kappa < \cardz$ such that for  all $\alpha \geq \kappa$,
    $A^{\alpha} = A^{\kappa}$ and $B^{\alpha}(a) =  B^{\kappa}(a)$ for all $a
    \in A^{\alpha}$. \qed 
  \end{enumerate}
\end{lemma}
\begin{proof}
  \begin{enumerate}[(i)]
  \item Induction over $\alpha$.

\notIncludedProof{  %todo
  \begin{itemize}
  \item If $\alpha = 0$, then $A^{\alpha} =  \text{least fixed point of }
       \lambda X\,.\, \sem{\ArgA}(\gamma_A, X, \lambda x\,.\,\emptyset)$.
 % \item If $\alpha = \beta + 1$
 % \item If $\alpha = \lambda$ limit
  \end{itemize}
}
  \item Induction over $\alpha$. %todo
  \item Let $\kappa$ be a regular cardinal of cardinality greater than
    that of all index sets which starts an inductive argument. By the
    inaccessibility of $\cardz$, $\kappa < \cardz$.  %todo
\qedhere
  \end{enumerate}
\end{proof}

\begin{theorem}
  $\sem{A_{\gamma_A, \gamma_B}} \in \sem{\Set}$ and $\sem{B_{\gamma_A,
      \gamma_B}} : \sem{A_{\gamma_A, \gamma_B}} \to \sem{\Set}$.
\end{theorem}
\begin{proof}
  By Lemma~\ref{thm:A-in-Set}, $\sem{A_{\gamma_A, \gamma_B}} = A^{\cardz} = A^{\kappa} \in
  \sem{\Set}$,  and similarly for $\sem{B_{\gamma_A, \gamma_B}}$. \qedhere
\end{proof}



\section{Remarks} %todo: better title
\label{sec:remarks}

\subsection{The r\^{o}le of large elimination}
\label{sec:large-elim}



\subsection{Indexed inductive definitions and induction-induction}
\label{sec:indexed-induction}



%\section{Conclusions}
%\label{sec:conclusions}

\bibliographystyle{alpha}
\bibliography{biblio}


\end{document}

